\section{Ergo Block Structure}
Ergo block consists of 4 parts:

\begin{itemize}
    \item{\em Header } - minimal amount of data, required to synchronize the chain and check PoW correctness.
    Also contains hashes of other sections.
    \item{\em BlockTransactions } - sequence of transactions, included in this block.
    \item{\em ADProofs } - proofs for transactions, included into BlockTransactions.
    Allows light clients to verify all the transactions and calculate new root hash.
    \item{\em Extension } - additional data, that does not correspond to previous sections.
          May contain interlinks, current miner votes and (sometimes) current parameters of the chain.
\end{itemize}

\subsection{Header}
\vspace{1em}
\begin{tabular}{ |p{2.5cm}||p{0.5cm}|p{7.5cm}|  }
    \hline
    \hline
    Field & Size & Description  \\
    \hline
    version  &  1 &  block version, to be increased on every soft- and hardfork  \\
    \hline
    parentId &  32 &  id of parent block  \\
    \hline
    ADProofsRoot &  32 &  hash of ADProofs for transactions in a block \\
    \hline
    stateRoot &  32 &  root hash (for an AVL+ tree) of a state after block application  \\
    \hline
    transactionsRoot  &  32 &  root hash (for a Merkle tree) of transactions in a block  \\
    \hline
    timestamp &  8 &  block timestamp(in milliseconds since beginning of Unix Epoch)  \\
    \hline
    nBits &  8 & current difficulty in a compressed view  \\
    \hline
    height &  4 & block height  \\
    \hline
    extensionHash & 32 & root hash of extension section  \\
    \hline
    equihashSolution & 128 & solution of equihash PoW puzzle  \\
    \hline
\end{tabular}

\vspace{1em}
Some of this fields may be calculated by node by itself if it is in correct mode:

\begin{itemize}
    \item parentId: if(status==bootstrap AND PoPoWBootstrap == false).
    \item interlinksRoot: if(PoPoWBootstrap == false).
    \item ADProofsRoot: if(status==regular AND ADState==false AND BlocksToKeep>0).
    \item stateRoot: if(status==regular AND ADState==false AND BlocksToKeep>0).
\end{itemize}

\subsection{Extension}

Extension is a key-value storage for a variety of data.
It contains 2 parts:
\begin{itemize}
    \item{\em Mandatory fields } - fields, that are known to all node in the network and may be changed
    via soft/hard forks only. This fields have 4 bytes key and at most 64 bytes value.
    Value length is known for all peers and this limit only important due to soft forks -
    it is allowed to add at most 1 new key to this section per epoch (256 blocks between
    difficulty recalculation), if key is not known to a peer it assumes that soft fork that
    added this key was performed.
    \item{\em Optional fields } - random data miner may add to a block. This section contains at most 8
    elements with 32 byte key size and at most 1024 bytes value size.
\end{itemize}

