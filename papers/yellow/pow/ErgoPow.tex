\documentclass[]{article}
\RequirePackage{amsmath}

\usepackage{graphicx}
%\graphicspath{{./figures/}}
\usepackage{amssymb}
\usepackage{color}
\usepackage{hyperref}
\usepackage{algorithm}
\usepackage{algpseudocode}
\bibliographystyle{IEEEtran}

\newcommand{\knote}[1]{\textcolor{green}{A: {#1}}}
\newcommand{\dnote}[1]{\textcolor{red}{D: {#1}}}
\newcommand{\vk}[1]{\textcolor{blue}{V: {#1}}}
\newcommand{\lnote}[1]{\textcolor{cyan}{L: {#1}}}
\newcommand{\Name}{$Autolykos$}
\def\Let#1#2{\State #1 $:=$ #2}
\def\LetRnd#1#2{\State #1 $\gets$ #2}

\newcommand{\pk}{\mathsf{pk}}
\newcommand{\sk}{\mathsf{sk}}

\begin{document}
    \title{\Name: The Ergo Platform PoW Puzzle}

    \author{Alexander Chepurnoy, Vasily Kharin, Dmitry Meshkov}

    \date{April 6, 2019\\v1.0}
    \maketitle

    %    \begin{abstract}
    %        This document contains the full description of \Name~-- the PoW protocol that is going to be used in Ergo platform.
    %    \end{abstract}


    \section{Introduction}

    Security of Proof-of-Work blockchains relies
    on multiple miners trying to produce new blocks by
    participating in PoW puzzle lottery, and the network is secure if the
    majority of them are honest. However, the reality becomes much more complicated
    than the original one-CPU-one-vote idea from the Bitcoin whitepaper\cite{nakamoto2008bitcoin}.

    The first threat to decentralization came from mining pools -- miners tend
    to unite in mining pools.
    Regardless of the PoW algorithm number of pools controlling more then 50\% of
    computational power is usually quite small: 4 pools in Bitcoin, 2 in Ethereum, 3 in ZCash, etc.
    This problem led to the notion of non-outsourceable puzzles~\cite{miller2015nonoutsourceable,daian2017piecework}.
    These are the puzzles constructed in such a way that if a mining pool outsources the puzzle
    to a miner, miner can recover pool's private key and steal the reward with a non-negligible probability.
    However the existing solutions either have too large solution size (kilobyte is already
    on the edge of acceptability for distributed ledgers) or very specific and
    can not be modified or extended in any way without breaking non-outsourceability.

    The second threat to cryptocurrencies decentralization is that ASIC-equipped miners are
    able to find PoW solutions orders of magnitude faster and more efficiently
    than miners equipped with the commodity hardware. In order to reduce the
    disparity between the ASICs and regular hardware, memory-bound computations
    where proposed in~\cite{dwork2003memory}. The most interesting practical
    examples are two
    asymmetric memory-hard PoW schemes which require significantly less memory
    to verify a solution than to find it~\cite{biryukov2017equihash,ethHash}.
    Despite the fact that ASICs already exist for both of them~\cite{ETHAsics,EquihashAsics},
    they remain the only asymmetric memory-hard PoW algorithms in use.

    In this paper we propose \Name{} --- new asymmetric memory-hard non-outsourceable PoW puzzle.
    In Section~\ref{puzzle} we provide a full
    specification of \Name, while in Section~\ref{discussion} we discuss its
    properties. Few auxiliary algorithms are placed in~\nameref{appendix}.

    \section{Ergo PoW puzzle}
    \label{puzzle}

    The proposed scheme requires following components:
    \begin{enumerate}
        \item Cyclic group $\mathbb{G}$ of prime order~$q$ with fixed generator~$g$
        and identity element~$e$.
        Secp256k1 elliptic curve is used for this purposes.
        \item Number of elements $k$ required in the solution. Value $k=32$ is used in
        implementation.
        \item Number $N$ of elements in the list
        $R\subset\mathbb{Z}/q\mathbb{Z}$ to be stored in miner's memory.
        Value $N=2^{26}$ is used in implementation.
        \item Hash function $H$ which returns the values in $\mathbb{Z}/q\mathbb{Z}$.
        Particular implementation is based on Blake2b256 and is described in Alg.\ref{alg:H}.
        \item Hash function $genIndexes$ which returns a list of numbers from
        $0\dots(N-1)$ of size $k$.
        It is based on Blake2b256 and is described in Alg.\ref{alg:genIndexes}.
        \item Target interval parameter $b$, that is recalculated via difficulty adjustment rules.
        \item Constant message $M=[0,\dots,1023].flatMap(i => Longs.toByteArray(i))$ that is used to enlarge message size and increase elements calculation time.
    \end{enumerate}

    \Name{} is based on one list $k$-sum problem: miner should find
    $k$ elements from the pre-defined list $R$ of size $N$, such that
    $\sum_{j \in J} r_{j} - sk = d$ is in the interval $\{-b,\dots,0,\dots,b\mod q\}$.
    In addition, we require set of element indexes $J$ to be obtained
    by one-way pseudo-random function $genIndexes$. This prevents optimizations as
    soon as it is hard to find such a seed,
    that $genIndexes(seed)$ returns the desired indexes.

    Thus we assume that the only option for miner is to use the simple brute-force algorithm~\ref{alg:prove} to
    create a valid block.

    \begin{algorithm}[H]
        \caption{Block mining}
        \label{alg:prove}
        \begin{algorithmic}[1]
            \State \textbf{Input}: upcoming block header hash $m$, key pair $pk=g^{sk}$
            \State Generate randomly a new key pair $w=g^x$
            \State Calculate $r_{i \in [0,N)}=H(j||M||pk||m||w)$
            \While{$true$}
            \LetRnd{$nonce$}{$\mathsf{rand}$}
            \Let{$J$}{$genIndexes(m||nonce)$}
            \Let{$d$}{$\sum_{j \in J}{r_j} \cdot x - sk \mod q$}
            \If{$d < b$}
            \State \Return $(m,pk,w,nonce,d)$
            \EndIf
            \EndWhile
        \end{algorithmic}
    \end{algorithm}

    Note that although the mining process utilizes private keys, solution itself
    only contains public keys. Solution verification can be performed by Alg.~\ref{alg:verify}.

    \begin{algorithm}[H]
        \caption{Solution verification}
        \label{alg:verify}
        \begin{algorithmic}[1]
            \State \textbf{Input}: $m,pk,w,nonce,d$
            \State require $d < b$
            \State require $pk,w\in \mathbb{G}$ and $pk,w \ne e$
            \Let{$J$}{$genIndexes(m||nonce)$}
            \Let{$f$}{$\sum_{j \in J} H(j||M||pk||m||w)$}
            \State require $w^f = g^dpk$
        \end{algorithmic}
    \end{algorithm}


    \section{Discussion}
    \label{discussion}

    First, notice that in Algorithm~\ref{alg:prove} we refer to construction
    $f(m,nonce,w,pk)=\sum_{j\in genIndexes(m||nonce)} H(j||M||pk||w)$ as a hash
    function. Public key plays a role of commitment. Therefore, the pair
    $(pk,d)$ is a Schnorr signature with a public key $w$ over the message
    $(m,nonce)$ with a hash function $f$. If one denotes $e$ the corresponding
    value of $f$, and pass to more common notations:
    $e=f(m,nonce,w,w^eg^{-d})$. The puzzle consists in trying
    different nonces and keys in order for signature to satisfy
    $d\in\{-b,\dots,0,\dots,b\}$. Security follows from the security of Schnor
    signatures, and outsourcing the puzzle is equivalent to outsourcing the
    signature (or parts of signature creation routine). The only difference from
    conventional setup is the design of function $f$. It must be constructed in such
    a way that efficient massive evaluations with different nonces require
    allocating large amount of memory (benefitting from data reuse), whereas
    single evaluation on verifier's side can be done ``on fly''.

    To achieve this, algorithm~\ref{alg:prove} requires to keep the whole list $R$ during the main loop.
    Every pre-calculated hash occupies 32 bytes, so the whole list of $N$ elements
    occupies $N \cdot 32 = 2 Gb$ of memory.
    For sure, a miner can recalculate these elements ``on fly'' during the main loop and
    thus reduce memory requirements.
    However in such a case the number of calls of $H$ will significantly grow up~(e.g.
    assuming GPU hashrate $G = 2^{30} H/s$~\cite{gpuHashrate} and block interval $t=120~s$,
    every element will be used $(G / N) \cdot k \cdot t = 3 \cdot 10^4$ times on average.)
    reducing miner's efficiency and profit.

    While list $R$ is quite big, it's filling consumes quite a lot of time: our initial implementation~\cite{ergoMiner}
    consumed ~25 seconds on Nvidia GTX 1070 to fill list $R$. This part, however, may be sufficiently
    optimized if miner in addition stores a list of unfinalized hashes $u_{i \in [0,N)}=H(i||M||pk$
    in memory, consuming 5 more Gigabytes of it. In such a case this work to calculate unfinalized hashes should
    be done only once during mining initialization while finalizing them and filling the list $R$
    for the new header will only consume few milliseconds~(for about 50 ms on Nvidia GTX 1070).

    The protocol is quite efficient in terms of solution size and verification time: it consists of 2 public keys of
    size 32 bytes, number $d$ that is at most 32 bytes
    (but contains a lot of leading zeros in case of the small target $b$) and an
    8-bytes long nonce. Header verification requires verifier to calculate 1 $genIndexes$
    hash, $k$ hashes $H$ and perform two exponentiations in the group. Reference
    Scala implementation~\cite{ergoGit} allows verifying block header in 2
    milliseconds on Intel Core i5-7200U, 2.5GHz.

    \bibliography{references}

    \section*{Appendix}
    \label{appendix}

    Implementation of hash function $H$ which returns the values in $\mathbb{Z}/q\mathbb{Z}$:

    \begin{algorithm}[H]
        \caption{Numeric hash}
        \label{alg:H}
        \begin{algorithmic}[1]
            \Function{H}{$input$}
            \Let{$validRange$}{$(2^{256} / q) \cdot q$}
            \Let{$hashed$}{$Blake2b256(input)$}
            \If{$hashed < validRange$}
            \State \Return $hashed.mod(q)$
            \Else
            \State \Return $H(hashed)$
            \EndIf
            \EndFunction
        \end{algorithmic}
    \end{algorithm}

    Implementation of hash function $genIndexes$ which returns a list of size $k$ with numbers in $0\dots (N-1)$:

    \begin{algorithm}[H]
        \caption{Index generator}
        \label{alg:genIndexes}
        \begin{algorithmic}[1]
            \Function{genIndexes}{$seed$}
            \Let{$hash$}{$Blake2b256(seed)$}
            \Let{$extendedHash$}{$hash||hash$}
            \State \Return $(0\dots{k-1}).map(i => extendedHash.slice(i,i+4).mod(N))$
            \EndFunction
        \end{algorithmic}
    \end{algorithm}

\end{document}