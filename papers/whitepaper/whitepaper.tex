\documentclass[]{article}
\RequirePackage{amsmath}

\usepackage{graphicx}
\usepackage{amssymb}
\usepackage{color}
\usepackage{hyperref}
\usepackage{float}
\usepackage{algorithm}
\usepackage{algpseudocode}

\bibliographystyle{IEEEtran}

\newcommand{\knote}[1]{{\textcolor{green}{Alex notes}}{#1}}
\newcommand{\dnote}[1]{{\textcolor{red}{DNotes:{#1}}}}
\newcommand{\Ergo}{Ergo}
\newcommand{\Erg}{\textbf{Erg}}
\newcommand{\nanoErg}{\textbf{nanoErg}}
\def\Let#1#2{\State #1 $:=$ #2}
\def\LetRnd#1#2{\State #1 $\gets$ #2}

\begin{document}
    \title{Ergo: The Resilient Platform For Contractual Money}
    \author{Ergo Developers}

    %    \date{January 25, 2019\\v1.0}

    \maketitle

    \begin{abstract}
        \dnote{todo. This is detailed overview of Ergo platform for average-technical audience. Full technical description will be provided in yellowpaper (with a link from this whitepaper)}
    \end{abstract}

    %    #######################################################################################
    %    Базовые мысли:
    %
    %    Основные фичи - децентрализация и контракционные деньги
    %    Изначальные пользователи - фанаты децентрализации кто уже более-менее в теме, и устали от проблем BTC/ETH (пример - Justin, mx)
    %
    %    #######################################################################################


    \section{Introduction}

    %    Блокчейн дает безопасность (resilient), которая следует из децентрализации (на разных уровнях - майнеры, разработчики, легкие клиенты). (Bitcoin and its problems?)
    %    Безопасность идет высокой ценой (синхронизация и процессинг данных на куче компов), поэтому основное применение - денежное (Money)
    %    Для активного применения (platform) нужна адаптивность под окружение (голосование, выживаемость и т.д.) и возможности написания финансовых контрактов (Contractual)
    %    Вот это все в Эрго, лучше чем в Эфириуме и Биткоине.

    Started more ten years ago with the Bitcoin~\cite{nakamoto2008bitcoin}, blockchain technology proved to be a secure way on maintaining
    a public transaction ledger.
    Even after achieving a market capitalisation over \$300bn in 2017~\cite{btcPrice}
    no serious attacks were performed to the Bitcoin network\dnote{check} despite the high yield of a successful attack.
    At the same time, centralized services are continuously fails to resist attacks, regardless of a maintaining company size and expertize.
    \dnote{examples of attacks to crypto exchanges (mt. gox?), big companies (Facebook? Or better some bank), malicious or incompetent behaviour of some human.
    Discuss a bit more if there are good examples especially of the last point}.
    This outstanding resilience of cryptocurrencies is achieved by a combination of modern cryptographic algorithms like digital
    signatures, that are widely adapted outside cryptocurrencies space, and decentralization, that is ensures that only
    the majority of the network participants can affect its security.

    However, this resilience does not come for free.
    To use a blockchain without any trust, it's participants should download and process all the transactions in
    the network, utilizing network resources.
    Besides network utilization, transaction processing requires to spend some computational resources,
    especially if transactional language is flexible enough.
    Finally, blockchains participants should keep quite big amount of data in their local storages and
    this storage requirements are growing fast.
    Thus thousands of computers all over the world utilize various resources
    increasing the cost of single transaction processing,
    and users pay transaction fees~\cite{chepurnoy2018systematic} to prevent spam attacks and
    to encourage participation in the protocol.
    These fees may be very high during the high load~\cite{bitcoinFees},
    and ten years later blockchain technology is still mainly used in a money application, where the advantage of
    high security outweighs the disadvantage of high transaction costs.

    Besides of a plain currency usage, most of blockchains are used to build decentralized applications on top of them.
    Such applications utilize the ability to write smart contracts~\cite{szabo1994smart}, that implements their logic
    by means of blockchain-specific programming language.
    Ways to write smart contracts depends on cryptocurrency transactional model~\cite{zahnentferner2018chimeric} ---
    in UTXO-based cryptocurrencies like Bitcoin~\cite{nakamoto2008bitcoin} every coin is protected by a used-defined script,
    while in account-based cryptocurrencies like Ethereum~\cite{ethWhitepaper} transactions call a code from special contract accounts.
    While this models are equivalent \dnote{correct word here?}, protecting scripts implementation is not so natural for
    account model~(e.g. simple multi-signature contract bug led to \$150 million loss in 2017 in Ethereum~\cite{parityLock}),
    while arbitrary Turing-complete logic implementation may be tricky in UTXO model~(e.g. check this example of
    simple Turing-complete system implementation in \Ergo{}~\cite{chepurnoy2018self}).

    While contractual component is attractive in terms of building decentralized applications,
    it is also important that the blockchain will survive in a long term.
    For now the whole area is young and most of application-oriented blockchain platforms exists just for several years,
    while they already have known problems with performance degradation over time \dnote{links and may be more discussion of known problems}
    and their long-term survivability is questionable.
    This problem led to concepts of lightened nodes with minimum storage requirements~\cite{reyzin2017improving},
    storage rent fee component that prevents bloating of full-node requirements~\cite{chepurnoy2018systematic},
    self-amendable protocols that can adopt to changing environment and improve themselves without
    trusted parties~\cite{goodman2014tezos}, and more.

    The main goal of the Ergo platform is to solve these problems by implementing known scientific ideas.
    It may be called a ``blockchain 1.1'' implementation --- a major update to
    blockchain technology instead of revolutionary breaking changes.
    The objective of Ergo is to be the platform for blockchain-demanding decentralized
    applications, survivable in the long-term and reliable to external or internal threats.
    In the following sections we summarize the general design of Ergo, it's technical and economic solutions,
    as well as provide possible examples of it's use-cases and further development.

    \section{Social contract}
    \label{sec:social}

    \Ergo{} protocol is very flexible and may be changed in the future by the community.
    In this section we define the main principles, that should be followed during the \Ergo{} protocol updates.
    In case of intentional violation of any of these principles, the resulting protocol should not
    be called \Ergo{}.

    \begin{itemize}
        \item{\em Decentralization first.} \Ergo{} should be as decentralized as possible.
        This means that any parties (social leaders, software developers, hardware manufacturers, miners and so on)
        which absence or malicious behaviour may affect the security of the network should be avoided.
        In particular, this means that treasury funds~(see \ref{sec:currency}) should be distributed
        in decentralized manner after the initial 1 year period.
        \item{\em Created for regular people.} \Ergo{} is the platform for regular people, their interests should
        not be infringed upon in favor of big players. In particular, that means that they should be able to
        participate in the protocol by running a full node and mine blocks (with small probability).
        \item{\em Platform for Contractual money.} \Ergo{} is the base layer to applications, that will be
        build on top of it. It allows to implement any kind of applications, but it's main goal is
        to provide efficient, secure and easy way to implement financial contracts. \dnote{probably kushti may write this better}
        \item{\em Long terms focus.} All aspects of \Ergo{} development should be focus on long-term perspective.
        At any point of time, \Ergo{} should be able to survive for centuries without expected hardforks,
        software or hardware improvements or some other unpredictable changes. As far as \Ergo{} is oriented
        to be a platform, applications built on top of \Ergo{} should also be able to survive in a long term.
        \item{\em Permissionless and open.} \Ergo{} protocol does not restrict or limit any categories of usage.
        It should allow anyone to join the network and participate the protocol without any preliminary permissions.
        No bailouts, blacklists or other forms of discrimination should be possible on core level of \Ergo{} protocol.
        On the other hand application developers are free to implement any logic they want, taking responsibility
        for the ethics and legality of their application.
    \end{itemize}


    \section{Autolykos Consensus Protocol}
    \label{sec:autolykos}

    %   Почему выбрали PoW
    %   Известные проблемы PoW
    %   Детали Автоликуса

    The core component of any blockchain system is it's consensus protocol.
    Despite of an extensive research of possible alternatives to original the Proof-of-Work (PoW) protocol,
    it remains in demand due to simplicity, high security guaranties and friendliness to light clients.
    However, decade of extensive testing revealed several problems of the original one-CPU-one-vote idea.

    First known problem of PoW is that development of a specialized hardware (ASIC) allowed
    small group of ASIC-equipped miners to solve PoW puzzles orders of magnitude faster and more efficiently
    than everyone else. This problem can be solved with the help of memory-hard PoW schemes,
    that reduce the disparity between the ASICs and commodity hardware. Most promising approach here
    is to use asymmetric memory-hard PoW schemes that require significantly less memory
    to verify a solution than to find it where proposed~\cite{biryukov2017equihash,ethHash}.

    Second known threat to a PoW network decentralization, is that even big miners trend to unite in
    mining pools, leading to a situation when just few pool operators (5 in Bitcoin, 2 in Ethereum
    at the time of writing) controls more then 51\% of computational power.
    Although the problem has already been discussed in the community, no practical solutions were
    implementer before \Ergo{}.


    Ergo PoW protocol --- Autolykos --- is the first consensus protocol, that is both memory hard
    and pool resistant~\cite{Ergopow}.
    Autolykos is based on one list $k$-sum problem: miner should find
    $k=32$ elements from the pre-defined list $R$ of size $N=2^{26}$~(which have a size of 2 Gb),
    such that $\sum_{j \in J} r_{j} - sk = d$ is in the interval $\{-b,\dots,0,\dots,b\mod q\}$.
    Elements of list $R$ are obtained as a result of one-way computation from index $i$,
    two miner public keys $pk,w$ and hash of block header $m$ as $r_i=H(i||M||pk||m||w)$,
    where $M$ is a static big message that is used to make hash calculation slower.
    In addition, we require set of element indexes $J$ to be obtained
    by one-way pseudo-random function $genIndexes$, that prevents possible solutions
    search optimizations.

    Thus we assume that the only option for miner is to use the simple brute-force algorithm~\ref{alg:prove} to
    create a valid block.

    \begin{algorithm}[H]
        \caption{Block mining}
        \label{alg:prove}
        \begin{algorithmic}[1]
            \State \textbf{Input}: latest block header hahs $m$, key pair $pk=g^{sk}$
            \State Generate randomly a new key pair $w=g^x$
            \State Pre-calculate list $R$ where $r_i=H(j||M||pk||m||w)$
            \While{$true$}
            \LetRnd{$nonce$}{$\mathsf{rand}$}
            \Let{$J$}{$genIndexes(m||nonce)$}
            \Let{$d$}{$\sum_{j \in J}{r_j} \cdot x - sk \mod q$}
            \If{$d < b$}
            \State \Return $(m,pk,w,nonce,d)$
            \EndIf
            \EndWhile
        \end{algorithmic}
    \end{algorithm}

    Note that although the mining process utilizes private keys, solution itself
    only contains public keys. Solution verification can be performed by Alg.~\ref{alg:verify}.

    \begin{algorithm}[H]
        \caption{Solution verification}
        \label{alg:verify}
        \begin{algorithmic}[1]
            \State \textbf{Input}: $m,pk,w,nonce,d$
            \State require $d < b$
            \State require $pk,w\in \mathbb{G}$ and $pk,w \ne e$
            \Let{$J$}{$genIndexes(m||nonce)$}
            \Let{$f$}{$\sum_{j \in J} H(j||M||pk||m||w)$}
            \State require $w^f = g^dpk$
        \end{algorithmic}
    \end{algorithm}

    Target parameter $b$ is adjusted to the current network hashrate via difficulty adjustment
    algorithm~\cite{meshkov2017short}, that is trying to predict an upcoming 1024 block length
    epoch hashrate based on data from previous 8 epochs.
    Full technical specification of Autolykos may be found at~\cite{Ergopow}.

    \section{Survivability}
    \label{sec:survivability}

    %   well-tested solutions
    %   voting
    %   soft-forkability
    %   storage rent
    %   light clients (Пробелмы пользователей без легких клиентов, Дайджест узлы, nipopow/flight clients)

    Being a platform for contractual money, Ergo should also support long-term contracts for a
    period of a person's life.
    While even young project are experiencing issues with the performance degradation and
    adaptability to external conditions, leading to a situation when a decentralized cryptocurrency
    depends on a small group of developers, that should provide a hard-fork, otherwise the cryptocurrency
    won't survive.

    First common issue is that in pursuit of popularity blockchain developers implement ad-hoc
    solutions without proper preliminary research and testing.
    Such solutions inevitably lead to bugs, and regardless of whether they were exploited or not
    to bug fixes in a centralized manner, that makes the network even less secure.
    Ergo approach here is to use stable well-tested solutions, even if that leads to slower
    short-term innovation applicability.
    Most of Ergo solutions are formalized in scientific papers, presented at peer-reviewed conferences
    and are widely discussed in community.

%    storage fee should be here, as soon as we use it in the next section as an example

    Another important aspect of survivability, is that the environment changes and a blockchain should
    adopt to changing hardware infrastructure, appearing ideas that may improve security or
    scalability, arising use-cases and so on.
    If all the rules are fixed without any ability to change them in a decentralized manner, even
    simple constant change may lead to huge debates and community split, e.g. discussion of a block
    size limit in Bitcoin led to the network split into several independent coins.

    In contrast, Ergo protocols is self-amendable and is able to adopt to changing environment.
    In Ergo parameters like block size can be changed on-the-fly via miners voting.
    At the beginning of a 1024 blocks length voting epoch miner is proposing changes~(up to 2 parameters,
    e.g. to increase block size and to decrease storage fee factor) and during the rest of epoch miners
    vote, whether to approve these changes or not.
    If majority of votes within an epoch are supporting some~(or both) of these changes, a new value of the
    parameter should be written into the extension section of the first block of the next epoch and the
    network starts to use this update parameter value during block mining and validation.

    To absorb more fundamental changes, Ergo is following the approach of soft-forkability, that
    allows to change protocol significantly but keeping old nodes operating.
    At the beginning of an epoch, miner can also propose to vote for a fundamental change~(e.g.~to
    add new instruction to ErgoScript), describing affected validation rules.
    Voting for such breaking changes continues for 32768 blocks and requires at least $90\%$ of
    "Yes" votes to be accepted.
    Once being accepted, 32768 blocks length activation period started to give time to outdated
    nodes to update their software version, and after that changes are activated.
    If a node is still not updated after the activation period, it skips the specified checks,
    but continues to validate all the known rules.
    List of previous soft-fork changes is record into the extension to allow light nodes of
    any software version to join the network and read validation rules, it should not check.




    \section{Currency And Emission}
    \label{sec:currency}

    %    подробная информация об эмиссии, скрипту эмиссии, и всему-всему что нужно для фаундейшена

    The native currency of \Ergo{} platform is \Erg{} token, which unique property is
    that it is the only currency to pay storage rent in \Ergo{}~(see~\ref{sec:economy} for more details).
    One \Erg{} token is divisible to up to $10^9$ parts labeled as \nanoErg{}.

    All \Erg{} tokens that will ever be circulate in the system are presented in the
    initial state and are divided in 3 parts~(boxes):

    \begin{itemize}
        \item{\em No premine proof.} This box contain exactly one~\Erg{} and is protected by the script,
        that is preventing it from spending by anyone.
        Thus, it is a long-living box, that will live in the system until storage-rent component will
        destroy it.
        It's main purpose is to prove, that \Ergo{} mining was not started privately by anyone before
        the declared launch date.
        To achieve this, additional registers of this box contains news from media (Guardian, Xinhua, Vedomosti)
        and block ids from already established cryptocurrencies (Bitcoin and Ethereum).
        Thus, \Ergo{} mining could not be started before certain events in the real world and in
        cryptocurrency space.

        \item{\em Treasury.} This box contain 4330791.5 \Erg{} that will be used to fund \Ergo{}
        development.
        It's protecting script~\cite{link to corresponding ergo tree} consists of two parts.

        First, it ensures, that only predefined portion of the box value was taken.
        During blocks 1-525599 (2 years) 7.5 \Erg{} will be released every block,
        during blocks 525600-590399 (3 month) 4.5 \Erg{} will be released every block and finally
        during blocks 590400-655199 (3 month) 1.5 \Erg{} will be released every block.
        This rule ensures the presence of funds for \Ergo{} development for at least 2.5 years and
        prevents an excessive wealth concentration under control of small group of people, as soon
        as at any point of time it won't exceed 10\% of total number of coins in circulation.

        Second, it have a custom protection from unexpected spending.
        Initially it requires that spending transaction should be signed by at least 2 of 3 secret keys,
        that are under control of initial team members. When they spend this box, they are free to
        change this part of the script as they wish, for example by adding new members to protect foundation
        funds or switching to threshold by special token ownership~\cite{link to ...}.

        During the first year this funds will be used to cover pre-issued EFYT token~\cite{our website},
        after that they will be distributed in decentralized manner via community voting.


        \item{\em Miners reward.} This box contain 93409132 \Erg{} that will be collected by block miners
        as a reward of their work.
        It's protecting script~\cite{link to corresponding ergo tree} have several conditions.

        First it checks, that block reward is going to the exact miner, that solved the PoW puzzle, and
        may be spent not earlier than 720 blocks after the current block.
        This restrictions are done to prevent mining pools formation, see~\ref{sec:autolykos} for more details.

        Second, it ensures, that only predefined portion of the box value was taken.
        During blocks 1 - 525599 (2 years) miner will be able to collect 67.5 \Erg{} from this box,
        during blocks 525600 - 590399 (3 month) miner will be able to collect 66 \Erg{} and after
        that block reward will be reduced for 3 \Erg{} every 64800 blocks (3 months) until it will reach zero
        at block 2080799, that should happen roughly 8 years after the genesis block.


    \end{itemize}

    All these rules results in the following curve of efficient number of coins in circulation with time:

    \dnote{TODO: plot the correct emission curve - number of \Erg{} in circulation vs time. Miners and foundation part, like ZCash https://z.cash/blog/funding/}

    

\section{Contractual Money}
    \label{sec:contractual}

 In our opinion, the overwhelming majority of use-cases for a public blockchains (even those that claim to provide a general-purpose decentralized world computer) are for financial applications, which do not require Turing-completeness. For instance, if an oracle writes down non-financial data into the blockchain~(such as temperature), this data is usually used further in a financial
 contract. Another trivial observation we make is that many applications use digital tokens with mechanics different from the native token.

For an application developer, the Ergo Platform offers custom tokens~(which are first-class citizens) and a domain-specific language for writing box protecting
 conditions in order to implement flexible and secure financial applications.
 Ergo applications are defined in terms of protecting scripts built into boxes, which may also contain data involved in the execution.
 We use the term {\em contractual money} to define Ergs (and secondary tokens) whose usage is bounded by a contract. This applies to all tokens on the platform in existance because any box with its contents~(Ergs, tokens, data) is bounded by a contract.
 
 However, we can distinguish between two types of contractual Ergs. The first, called {\em free Ergs}, are the ones that could change their contracts easily and have no restrictions on the outputs or the other inputs of a spending transaction. The second type are {\em bounded Ergs}, whose contracts require the spending transaction to have input and output boxes with specific properties. 
 
 For example, if a box $A$ is protected by just a public key~(so providing a signature against a spending transaction is enough in order to destroy the box), the public key owner can spend $A$ and transfer the Ergs to any arbitrary output box. Thus, the Ergs within $A$ are free. 
% to change the contract. 
In contrast, imagine a box $B$ protected by combination of a public key and a condition that demands the spending transaction to create an output box with the same amount of Ergs as in $B$ and whose guarding script has the hash \texttt{rBMUEMuPQUx3GzgFZSsHmLMBouLabNZ4HcERm4N} (in Base58 encoding). In this case, the Ergs in $B$ are bounded Ergs.
 
 Similarly, we can define free and bounded tokens. An Ergo contract can have several hybrids such as bounded Ergs and free tokens or both bounded under one public key and free under another.

\subsection{Preliminaries For Ergo Contracts}

  While in Bitcoin, a transaction output is protected by a program in a stack-based language named {\em Script}, in Ergo a box is protected by a logic formula which combines predicates over a context with cryptographic statements provable via zero-knowledge protocols using AND, OR, and $k$-out-of-$n$ connectives. The formula is represented as a typed direct
 acyclic graph, whose serialized form is written in a box. To destroy a box, a spending transaction needs to provide arguments (which include zero-knowledge proofs) satisfying the formula.

 However, in most cases, a developer is unlikely to develop contracts in terms of graphs. Instead, he would like to use a high-level language such as ErgoScript, which we provide with the reference client. 
 
 Writing scripts in ErgoScript is easy. As an
 example, for a one-out-of-two signature, the protecting script would be ${pk_1 \|pk_2}$, which means ``prove knowledge of
 a secret key corresponding to the public key $pk_1$ or knowledge of a secret key corresponding to public key $pk_2$''. We have
 two separate documents for help in developing contracts with ErgoScript: the ``ErgoScript Tutorial''~\cite{ergoTutorial}
 and the ``Advanced ErgoScript Tutorial''~\cite{ergoAdvTutorial}. Thus, we do not get into the details of developing contracts with ErgoScript. Rather, we provide a couple of motivating examples in the following sections.

Two more features of Ergo shaping contracting possibilities are:

 \begin{itemize}
    \item {\em Data Inputs: }
 To be used in a transaction, a box need not be destroyed but can instead be read-only. In the latter case, we refer to the box as being part of the {\em data input} of the transaction. Thus, a transaction gets two box sets as its arguments, the inputs and
 data inputs, and produces one box set named {\em outputs}. Data inputs are useful for oracle applications and interacting contracts.

    \item {\em Custom Tokens: }
 A transaction can carry many tokens as long as the estimated complexity for processing them does not exceed a limit, a parameter that is set by miner voting. A transaction can also issue a single token with a unique identifier which is equal to identifier of a first~(spendable) input box of the transaction. The identifier is unique assuming the collision resistance of an underlying hash function.
 The amount of the tokens issued could be any number within the range $[1, 9223372036854775807]$. The weak preservation rule is followed for tokens, which requires that the total amount of any token in a transaction's outputs should be no more
 than total amount of that token in the transaction's inputs~(i.e., some amount of token could be burnt). In contrast, the strong reservation rule is followed for Ergs, which requires that the total amount of Ergs in the inputs and outputs must be same.
 \end{itemize}

\subsection{Contract Examples}
\label{sec:examples}

 In this section we provide some examples which demonstrate the superiority of Ergo contracts compared to Bitcoin's. The examples include betting on oracle-provided data, non-interactive mixing, atomic swaps, complementary currency, and an initial coin offering implemented on top of the Ergo blockchain.

 \subsubsection{An Oracle Example}
 \label{sec:platform}

 Equipped with custom tokens and data inputs, we can develop a simple oracle example which also shows some design patterns that we discovered while playing with Ergo contracts. Assume that Alice and Bob want to bet on tomorrow's weather by putting money into a box that becomes spendable by Alice if tomorrow's temperature is more than 15 degrees, and spendable by Bob otherwise. To deliver the temperature into the blockchain, a trusted oracle is needed.

 In contract to Ethereum with its long-lived accounts, where a trusted oracle's identifier is usually known in advance, delivering data with one-time boxes is more tricky. For starters, a box protected by the oracle's key cannot be trusted, as anyone can create such a box. It is possible to include signed data into a box and check the oracle's signature in the contract (we have such an example), but this is quite involved. Instead, a solution with custom
 tokens is very simple.

 Firstly, a token identifying the oracle should be issued. In the simplest case, the amount of this token could be one. We call such a token {\em a singleton token}. The oracle creates a box containing this token along with its data (i.e., the temperature) in register $R_4$ and the UNIX epoch time in register $R_5$.
 In order to update the temperature, the oracle destroys its box and creating a new one with the updated temperature.

 Assume that Alice and Bob know the oracle's token identifier in advance. With this knowledge, they can jointly create a box with a contract that requires first (read-only) data input to contain the oracle's token. The contract extracts the temperature and time from the data input
 and decides who gets the payout. The code is as simple as following:

 \begin{algorithm}[H]
    \caption{Oracle Contract Example}
    \label{alg:oracle}
    \begin{algorithmic}[1]
        \State val dataInput = CONTEXT.dataInputs(0)
        \State val inReg = dataInput.R4[Long].get
        \State val inTime = dataInput.R5[Long].get
        \State val inToken = dataInput.tokens(0).\_1 == tokenId
        \State val okContractLogic = (inTime $>$ 1556089223) \&\&
        \State\hspace{\algorithmicindent}\hspace{\algorithmicindent} ((inReg $>$ 15L \&\& pkA) $||$ (inReg $\le$ 15L \&\& pkB))
        \State inToken \&\& okContractLogic
    \end{algorithmic}
 \end{algorithm}

 This contract shows how a singleton token could be used for authentification. As a possible alternative, the oracle
 can put the time and temperature into a box along with a signature on this data. However, this requires signature verification, which is more complex and expensive compared to
 the singleton token approach. Also, the contract shows how read-only data inputs could be useful for contracts which need to access data stored in some other box in the state. Without data inputs, an oracle must issue one spendable box for each
 pair of Alice and Bob. With data inputs, the oracle issues only a single box.

\subsubsection{A Mixing Example}
 \label{sec:platform}

 Privacy is important for a digital currency but implementing it can be costly or require a trusted setup. Thus, it is desirable to find cheaper way for coin mixing. As a first step towards that, we offer a non-interactive mixing protocol between two users Alice and Bob that works as follows:
 \begin{enumerate}
    \item{} Alice creates a box which demands the spending transaction to satisfy certain conditions. After that, Alice only listens to the blockchain, no any interaction with Bob is needed.
    \item{} Bob creates a transaction spending Alice's box along with one of his own and generating two outputs with identical script but different data. Each of Alice and Bob may spend only one of the two outputs but to an external observer the two outputs look indistinguishable and he cannot decide which output belongs to whom.
 \end{enumerate}

 For simplicity, we do not consider fee in the example. The idea of mixing is similar to non-interactive Diffie-Hellman key exchange. First, Alice generates a secret value $x$~(a huge number) and publishes the corresponding public value $gX = g^x$. She requires Bob to generate a secret number $y$, and to include into each output two
 values $c_1$, $c_2$, where one value is equal to $g^y$ and the other is equal to $g^{xy}$. Bob uses a random coin to choose meanings for $\{c_1, c_2\}$. Without access to the secrets, an external observer cannot guess with probability better than  $\frac{1}{2}$ whether $c_1$ is equal to $g^y$ or to $g^{xy}$. This is assuming that the cryptographic primitive we use has a certain property, that the Decision Diffie-Hellman (DDH) problem is hard. To destroy an output box, a proof should be given that either $y$ is known such that $c_2 = g^y$, or $x$ is known such that $c_2 = c_1^x$.
 The contract of Alice's box checks that $c_1$ and $c_2$ are well-formed. The code snippets for the Alice's coin and the mixing transaction's output are provided in Algorithms \ref{alg:alice} and \ref{alg:mixing-out} respectively. Since ErgoScript currently doesn't have support for proving knowledge of some $x$ such that $c_2 = {c_1}^x$ for arbitrary $c_1$,  we will prove a slightly longer statement that is supported, namely, proving knowledge of $x$ such that $gX = g^x$ and $c_2 = {c_1}^x$. This is called proveDHTuple.

 \begin{algorithm}[H]
    \caption{Alice's Input Script}
    \label{alg:alice}
    \begin{algorithmic}[1]
        \State val c1 = OUTPUTS(0).R4[GroupElement].get
        \State val c2 = OUTPUTS(0).R5[GroupElement].get
        \State
        \State OUTPUTS.size == 2 \&\&
        \State OUTPUTS(0).value == SELF.value \&\&
        \State OUTPUTS(1).value == SELF.value \&\&
        \State blake2b256(OUTPUTS(0).propositionBytes) == fullMixScriptHash \&\&
        \State blake2b256(OUTPUTS(1).propositionBytes) == fullMixScriptHash \&\&
        \State OUTPUTS(1).R4[GroupElement].get == c2 \&\&
        \State OUTPUTS(1).R5[GroupElement].get == c1 \&\& \{
        \State\hspace{\algorithmicindent}  proveDHTuple(g, gX, c1, c2) $||$
        \State\hspace{\algorithmicindent}  proveDHTuple(g, gX, c2, c1)
        \State \}
    \end{algorithmic}
 \end{algorithm}

 \begin{algorithm}[H]
    \caption{Mixing Transaction Output Script}
    \label{alg:mixing-out}
    \begin{algorithmic}[1]
        \State val c1 = SELF.R4[GroupElement].get
        \State val c2 = SELF.R5[GroupElement].get
        \State proveDlog(c2) $||$            // either c2 is $g^y$
        \State proveDHTuple(g, c1, gX, c2) // or c2 is $u^y = g^{xy}$
    \end{algorithmic}
 \end{algorithm}

 We refer the reader to \cite{ergoAdvTutorial} for a proof of indistinguishability of the outputs and details on why Alice and Bob can spend only their respective coins.


\subsubsection{More Examples}

 In this section, we briefly shed light on a few more examples along with links to the documents providing the details and code.

\paragraph{Atomic Swap}
Cross-chain atomic swap between Ergo and any blockchain that supports payment to either SHA-256 or Blake2b-256 hash preimages and time-locks can be done in a similar way to that proposed for Bitcoin~\cite{Nol13}. An Ergo alternative implementation is provided in~\cite{ergoTutorial}. As Ergo also has custom tokens, atomic exchange on the single Ergo block chain (Erg-to-token or token-to-token) is also possible. An implementation for this can also be found in~\cite{ergoTutorial}.

\paragraph{Crowdfunding}

 We consider the simplest crowdfunding scenario. In this example, a crowdfunding project with a known public key is considered successful if it can collect unspent outputs with total value not less than a certain amount before a certain height. A project backer creates an output box protected by the following statement: the box can be spent 
 if the spending transaction has the first output box protected by the project's key and amount no less than the target amount.
 Then the project can collect (in a single transaction) the biggest backer output boxes with total value not less than the target amount~(it is possible to collect up to ~22,000 outputs, which is
 enough even for a big crowdfunding campaign). For remaining outputs, it is possible to construct follow-up transactions. The code can be found in~\cite{ergoTutorial}.

\paragraph{The Local Exchange Trading System}

 Here we briefly demonstrate a Local Exchange Trading System (LETS) in Ergo. In such a system, a member of a community may issue community currency via personal debt. For example, if Alice with zero balance is buying something for $5$
 community tokens from Bob, whose balance is zero as well, her balance after the trade would be $-5$ tokens, and
 Bob's balance would be $5$ tokens. Then Bob can buy something using his $5$ tokens, for example, from Carol.
 Usually, in such systems, there is a limit on negative balances (to avoid free-riding).

 Since a digital community is vulnerable to Sybil attacks~(which also enable free-riding), some mechanism is needed to prevent such attacks where Sybil nodes create debts. 
 The simplest solution is to use a committee of trusted managers that approve new members of the community. A trust-less but more complex solution is to use security deposits made in Ergs. For simplicity, we consider the approach with the committee here.
 
 \snote{Give more details/references about Sybil attacks}

 This example contains two interacting contracts. A {\em management contract} maintains a list of community members, and a new member can be added if some a management condition is satisfied  (for example, a threshold
 signature is provided). A new member is associated with a box containing a token that identifies the member. This box, which contains the {\em member contract}, is protected by a special exchange script that requires the spending transaction to do a fair exchange.
 We skip the corresponding code, which can be found in a separate article~\cite{letsTutorial}.
 
 What this contract shows, in contrast to the previous example, is that instead of storing the members list, only a short digest of an authenticated AVL+ tree can be included in the box. This allows reduction in storage requirements for the state. A transaction doing lookup or modification of the member list should provide a proof for AVL+ tree lookup or modification operations. Thus, saving space in the state storage leads to bigger transactions, but this scalability problem is easier to solve.

\paragraph{Initial Coin Offering}

 We discuss an Initial Coin Offering (ICO) example that shows how multi-stage contracts can be created in Ergo. Like most ICOs, our example has three stages. In the first stage, the project raises money in Ergs. In the second stage, the project issues a new token, whose amount equals the number of nanoErgs raised in the first stage. In the third stage, the investors can withdraw issued tokens.

 Note that the first and third stages have many transactions on
 the blockchain, while a single transaction is enough for the second stage. Similar to the previous example, the ICO contract uses an AVL+ tree to store the list of (investor, amount) pairs. The complete code is available at~\cite{icoTutorial}.


\paragraph{More Examples}

 We have even more examples of Ergo applications in \cite{ergoTutorial, ergoAdvTutorial}. These examples include time-controlled emission, cold wallets contracts, rock-paper-scissors game, and many others.

    \section{Conclusions}

    \bibliography{references}

\end{document}
