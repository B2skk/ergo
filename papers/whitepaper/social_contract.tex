\section{Ergo Vision}
\label{sec:social}

%\knote{Ergo Vision And The Social Contract ?}
%\dnote{May be just Ergo Vision? }

The \Ergo{} protocol is very flexible and may be changed in the future by the community.
In this section we define the main principles that should be followed in Ergo which
might be referred to as ``Ergo's Social Contract''.
In case of intentional violation of any of these principles, the resulting protocol should not
be called \Ergo{}.


\begin{itemize}
    \item{\em Decentralization First.} \Ergo{} should be as decentralized as possible: any parties (social leaders, software developers, hardware manufacturers, miners, funds and so on)
    whose absence or malicious behavior may affect the security of the network should be avoided.
    If any of these parties do appear during \Ergo{}'s lifetime, the community should consider ways to decrease their impact level.
    \item{\em Created for Regular People.} \Ergo{} is a platform for ordinary people, and their interests should not be infringed upon in favor of big parties.
    In particular, this means that centralization of mining should be prevented and regular people should be able to participate in the protocol by running a full node and mining blocks (albeit with a small probability).
    \item{\em Platform for Contractual Money.} \Ergo{} is the base layer to applications that will be built on top of it.
    It is suitable for several applications but its main focus is to provide an efficient, secure and easy way to implement financial contracts.
    \item{\em Long-term Focus.} All aspects of \Ergo{} development should be focused on a long term perspective.
    At any point of time, \Ergo{} should be able to survive for centuries without expected hard forks,
    software or hardware improvements or some other unpredictable changes.
    Since Ergo is designed as a platform, applications built on top of Ergo should also be able to survive in the long term.
    This resiliency and long term survivability may also enable Ergo to be a good store of value.
    \item{\em Permissionless and Open.} \Ergo{} protocol does not restrict or limit any categories of usage.
    It should allow anyone to join the network and participate in the protocol without any preliminary actions.
    Unlike the traditional financial system, no bailouts, blacklists or other forms of discrimination should be possible
    on the core level of \Ergo{} protocol.
    On the other hand application developers are free to implement any logic they want, taking responsibility for the ethics and legality of their application.
\end{itemize}