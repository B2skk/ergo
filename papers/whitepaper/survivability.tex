\section{Resiliency and Survivability}
\label{sec:survivability}

\dnote{ rewrite in a more technical style?}

% well-tested solutions
% voting
% soft-forkability
% storage rent
% light clients (Пробелмы пользователей без легких клиентов, Дайджест узлы, nipopow/flight clients)

Being a platform for contractual money, Ergo should also support long-term contracts for a
period of a person's life.
At the same time, even young existing smart contract platforms are experiencing issues with performance degradation and
adaptability to external conditions.
This leads to a situation where a cryptocurrency depends on a small group of developers
that should provide a fixing hard-fork, or the cryptocurrency won't survive.

The first common survivability issue is that in pursuit of popularity blockchain developers implement ad-hoc
solutions without proper preliminary research and testing.
Such solutions inevitably lead to bugs, hasty bug fixes, fixes of bug fixes and so on, making the network even less secure.
Ergo approach here is to use stable well-tested solutions, even if that leads to slower
short-term innovations.
Most of Ergo solutions are formalized in papers presented at peer-reviewed conferences
and have been widely discussed in the community.

A second problem decentralization and thus survivability faces is lack of secure trustless light clients.
Ergo is trying to fix known problems of blockchain technology without creating new problems.
As far as Ergo is a PoW blockchain, it easily allows extracting a small header from the block content.
Header alone allows for validation of the work done on it, while headers-chain is enough for best chain selection
and synchronization with the network.
Headers-chain is much smaller than the full blockchain, however, it is still growing linearly with time.
Hopefully, modern research of light clients~\cite{kiayias2017non,luuflyclient} provide a way to
synchronize with the network by downloading even smaller amount of data, unlocking the ability to
use the network without any trust from low-end hardware like mobile phones.
Also, Ergo uses authenticated state~\cite{reyzin2017improving} and for transactions included
a client may download a proof of their correctness.
The proof is generated by a block miner and allows to check all the state transitions happened within the block, namely,
that transaction inputs not being spent before the transaction, and that the
transaction outputs has been indeed added to the state.
Thus, regardless of the blockchain size a regular user with
a mobile phone can join the network and start using Ergo with the same security
guarantees as a full node. \knote{Tell a story about light full-nodes as well.}
\dnote{If we're going to extract "Ergo state" section, this story should go there}

Readers may see a third potential problem in that although support of light clients
solves problems of Ergo users, it does not solve problems
of Ergo miners that still should keep the whole state for efficient
transaction validation.
In existing blockchain systems, users can put arbitrary data to this state forever,
creating a lot of dust in it and increasing its size over time~\cite{perez2019another}.
Big state size leads to serious security issues when the state does not fit in random-access memory,
an adversary may generate transactions which validation become very slow due to required random
access to miners storage
leading to DoS attack like an attack to the Ethereum network in 2016~\cite{??}.
Moreover the community`s fear of these attacks and the problem of ``state bloat'' without any sort of
compensation to miners and users holding the state may have prevented scaling solutions that otherwise
could have been implemented such as somewhat larger block sizes for example.
To prevent this, Ergo has a storage rent component: if an
output remains in the state for 4 years without being moved, a miner may charge a small fee for every
byte kept in the state.
This idea is similar to regular cloud storage services however, it was only proposed quite recently for
cryptocurrencies~\cite{chepurnoy2017space} and has several important consequences.
First, Ergo mining will always be stable, unlike Bitcoin and other PoW currencies,
in which mining may become unstable after emission done~\cite{carlsten2016instability}.
Second, state size growth becomes controllable and predictable reducing hardware requirements for Ergo miners.
Third, by collecting storage fees from outdated boxes, miners return coins to circulation preventing a steady decrease
of circulating supply due to lost keys~\cite{wsj2018}.
All these effects should support Ergo long-term survivability, both technically and economically.

A fourth vital challenge to survivability is that the environment changes, so the network should
adapt to ever changing hardware infrastructure, ideas that may improve security or
scalability emerge over time, use-cases evolution, and so on.
If all the rules are fixed without any ability to change them in a decentralized manner, even
simple constant change may lead to hot debates and community splits, e.g., discussion of a block
size limit in Bitcoin led to the network split into several independent coins.
In contrast, Ergo protocol is self-amendable and is able to adapt to the changing environment.
In Ergo, parameters like block size can be changed on-the-fly via miners voting.
At the beginning of a 1024-blocks voting epoch a miner is proposing changes~(up to 2 parameters,
e.g., to increase block size and to decrease storage fee factor) and during the rest of the epoch miners
vote to whether approve the changes or not.
If the majority of votes within an epoch are supporting a change, a new value of
the parameter should be written into the extension section of the first block of the next epoch, and
the network starts to use this update parameter value during block mining and validation.
\dnote{highlight what parameters can be voted by miners}

To absorb more fundamental changes, Ergo is following the approach of soft-forkability that
allows to change protocol significantly with keeping old nodes operating.
At the beginning of an epoch, a miner can also propose to vote for a fundamental change~(e.g., to add a new instruction to ErgoScript), describing affected validation rules.
Voting for such breaking changes continues for 32,768 blocks and requires for at least $90\%$ of
"Yes" votes to be accepted.
Once being accepted, 32,768 blocks long activation period starts to give outdated
nodes time to update their software version.
If a node software is still not updated after the activation period, then it skips the specified checks, but continues
to validate all the known rules.
List of previous soft-fork changes is recorded into the extension to allow light nodes of
any software version to join the network and catch up current validation rules.