\section{Introduction}
\label{sec:intro}

%    Блокчейн дает безопасность (resilient), которая следует из децентрализации (на разных уровнях - майнеры, разработчики, легкие клиенты). (Bitcoin and its problems?)
%    Безопасность идет высокой ценой (синхронизация и процессинг данных на куче компов), поэтому основное применение - денежное (Money)
%    Для активного применения (platform) нужна адаптивность под окружение (голосование, выживаемость и т.д.) и возможности написания финансовых контрактов (Contractual)
%    Вот это все в Эрго, лучше чем в Эфириуме и Биткоине.

Started more ten years ago with the Bitcoin~\cite{nakamoto2008bitcoin}, blockchain technology proved
to be a secure way of maintaining a public transaction ledger.
Even after achieving a market capitalization over \$300bn in 2017~\cite{btcPrice},
no severe attacks were performed to the Bitcoin network despite the high potential yield.
This outstanding resilience of cryptocurrencies is achieved by a combination of modern cryptographic algorithms
and decentralized architecture.
Cryptography itself is widely used outside the cryptocurrencies, however centralized services continuously
being hacked\cite{sanger2015bank,leskin2018top} due to malicious or incompetent behavior of company employers.
\knote{Why you are sure that this is a major problem of centralized services?}\dnote{I've provided 2 links, we may provide more if you don't think it is enough}
Decentralized systems are resilient for such kind of attacks by design due to absence of an ``admin'' user
that is capable to manually change records of the database and steal money or private data.

However, this resilience does not come for free.
To use a blockchain without any trust, its participants should check each other by downloading and
processing all the transactions in the network, utilizing network resources.
Besides network utilization, transaction processing requires to utilize computational resources,
especially if the transactional language is flexible enough.
Finally, blockchain participants should keep quite a significant amount of data in their local storages and
the storage requirements are growing fast.
Thus transaction processing utilize various resources of thousands of computers all over the world
and these resources consumption is paid by regular users via transaction fees~\cite{chepurnoy2018systematic}.
These fees may be very high~\cite{bitcoinFees}
and ten years later blockchain technology is still mainly used in financial applications, where the advantage of
high security outweighs the disadvantage of high transaction costs.

Besides of a plain currency usage, most of the blockchains are used to build decentralized applications on top of them.
Such applications utilize the ability to write smart contracts~\cite{szabo1994smart}, that implements their logic
by means of blockchain-specific programming language.
For now, there are two main approaches to write smart contracts~\cite{zahnentferner2018chimeric}:
UTXO-based~(e.g., Bitcoin) and account-based~(e.g., Ethereum).
Account-based cryptocurrencies, such as Ethereum, introduce special contract accounts controlled by code,
that may be invoked by incoming transactions.
This approach allows performing arbitrary computations, however, implementation of complex coins spending conditions
lead to bugs like \$150 million loss in 2017 in
Ethereum from simple multi-signature contract~\cite{parityLock}.
In UTXO-based cryptocurrencies, every coin has a script associated with it, and to spend the coin, one should
satisfy script conditions.
Implementation of coins protecting conditions is much easier in UTXO model,
while arbitrary Turing-complete logic become quite complicated~(e.g., implementation of a trivial Turing-complete
system become quite complicated~\cite{chepurnoy2018self}).
Ergo is based on UTXO model as far as it is more convenient to implement financial applications covering the
overwhelming majority of public blockchain use-cases.

While the contractual component is attractive in for building decentralized applications,
it is also essential that the blockchain will survive in the long term.
For now, the whole area is young, and most of the application-oriented blockchain platforms exist just for several years,
while they already have known problems with performance degradation over time~\cite{???} and their long-term survivability is questionable.
This problem led to concepts of light nodes with minimum storage requirements~\cite{reyzin2017improving},
storage rent fee component that prevents bloating of full-node requirements~\cite{chepurnoy2018systematic},
self-amendable protocols that can adapt to the changing environment and improve themselves without
trusted parties~\cite{goodman2014tezos}, and more.
Ergo combines various scientific ideas together to fix these problems, while also provide a way for
further improvements without any breaking changes.

This paper contains a detailed overview of the Ergo platform %for average-technical audience
and is structured as follows: in Section~\ref{sec:social} we present general ideas of Ergo platform,
Section~\ref{sec:autolykos} contains description of Ergo consensus protocol,
in Section~\ref{sec:survivability} we describe main components, that are implemented in Ergo
to achieve long term survivability, Section~\ref{sec:currency} describes native token
of Ergo platform and its emission rules while in Section~\ref{sec:contractual} we
discuss its value and possible use-cases of Ergo contracts.
Finally, we conclude our work in Section~\ref{sec:conclusions}.
More information about different Ergo components may be found in separate documents:
full technical documentation~\cite{yellowpaper}, ErgoScript overview and examples\cite{???,???} and
Proof-of-Work algorithm specification~\cite{Ergopow}.
